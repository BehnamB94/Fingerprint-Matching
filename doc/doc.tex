\documentclass[conference]{IEEEtran}
\IEEEoverridecommandlockouts
% The preceding line is only needed to identify funding in the first footnote. If that is unneeded, please comment it out.
\usepackage{cite}
\usepackage{amsmath,amssymb,amsfonts}
\usepackage{algorithmic}
\usepackage{graphicx}
\usepackage{textcomp}
\usepackage{xcolor}
%\def\BibTeX{{\rm B\kern-.05em{\sc i\kern-.025em b}\kern-.08em
%    T\kern-.1667em\lower.7ex\hbox{E}\kern-.125emX}}
\bibliographystyle{IEEEtran}

\begin{document}

\title{Fingerprint Matching Based on Convolutional Neural Networks}
\author{\IEEEauthorblockN{1\textsuperscript{st} Given Name Surname}
\IEEEauthorblockA{\textit{dept. name of organization (of Aff.)} \\
\textit{name of organization (of Aff.)}\\
City, Country \\
email address}
\and
\IEEEauthorblockN{2\textsuperscript{nd} Given Name Surname}
\IEEEauthorblockA{\textit{dept. name of organization (of Aff.)} \\
\textit{name of organization (of Aff.)}\\
City, Country \\
email address}
}

\maketitle

\begin{abstract}
Fingerprint has been widely used in biometric authentication systems due to its uniqueness and consistency. Despite tremendous progress made in automatic fingerprint identification systems (AFIS), highly efficient and accurate fingerprint matching remains a critical challenge. In this paper, we propose a novel fingerprint matching method based on Convolutional Neural Networks (ConvNets). The fingerprint matching problem is formulated as a classification system, in which an elaborately designed ConvNets is learned to classify each fingerprint pair as a match or not. A key contribution of this work is to directly learn relational features, which indicate identity similarities, from raw pixels of fingerprint pairs. In order to achieve robustness and characterize the similarities comprehensively, incomplete and partial fingerprint pairs were taken into account to extract complementary features. Experimental results on FVC2002 database demonstrate the high performance of the proposed method in terms of both false acceptance rate (FAR) and false rejection rate (FRR). Thanks to the robustness of feature extraction, the proposed method is applicable of incomplete and partial fingerprint matching.
\end{abstract}

\begin{IEEEkeywords}
fingerprint matching, convolutional neural networks
\end{IEEEkeywords}

\section{Introduction}
Introduction.

\section{Related Works}

here is the only related work \cite{mainRef}
\bibliography{refrences}
\end{document}
